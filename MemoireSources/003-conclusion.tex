\chapter*{Conclusion}
\addcontentsline{toc}{chapter}{Conclusion}
\markboth{conclusion}{conclusion}

Cette étude démontre que la combinaison de divers outils est un moyen de recueillir une multitude de données afin de révéler la partie invisible de la performance. \\

L'analyse de ces données contribue à la compréhension de la performance, qui est un aspect déterminant de la réussite.\\

En outre, le fait de pouvoir comprendre la performance peut permettre à l'entraîneur d'adapter les entraînements pour qu'ils correspondent à la physiologie de l'athlète et lui permettre de développer au maximum son potentiel. Par la suite, cela pourrait également répondre à un autre besoin qui est celui de prédire le pic de forme de l'athlète. \\

La présente étude est donc le point de départ de diverses autres recherches, car les besoins à traiter sont nombreux.\\

Pour conclure, nous pouvons aujourd'hui affirmer qu'aucune technologie, la plus avancée soit-elle, ne peut remplacer le talent du sportif, l'intelligence de l’entraîneur et les longues heures d’entraînement, mais la technologie peut aider les entraîneurs dans leurs analyses et les athlètes dans l'amélioration de leurs performances.\\






%%% Local Variables: 
%%% mode: latex
%%% TeX-master: "memoire"
%%% End: