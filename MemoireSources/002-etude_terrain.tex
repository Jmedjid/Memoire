\chapter*{Étude de terrain et délimitation du sujet}
\addcontentsline{toc}{chapter}{Étude de terrain}
\markboth{Étude de terrain}{Étude de terrain}
\label{chap:etude_terrain}



Le thème de ce mémoire a été très facile à trouver car le lien entre l'informatique et le sport était évident pour moi, cependant la problématique à quant à elle été compliquée à trouver. En effet, beaucoup de pistes étaient intéressantes à explorer, cependant aucune d'entre elles ne résolvaient un réel problème ou bien elles étaient trop compliquer à résoudre.\\

Mon tuteur, Mr Lom Hillah, m'a alors suggéré d'effectuer une étude de terrain auprès des athlètes de mon club d'athlétisme afin de connaître leurs besoins. 
Beaucoup d'entre eux m'ont fait part de leur souhait de pouvoir améliorer leurs performances grâce à l'utilisation de la technologie et notamment de divers outils informatiques. Cet axe était également une de mes premières idées, cependant en en faisant part à mon tuteur mais également à mon entraîneur, je me suis rendu compte qu'elle serait trop compliquée à étudier. En effet, pour pouvoir savoir si un outil permet d'améliorer la performance il faudrait tester cet outil sur un échantillon de plusieurs athlètes mais également toujours sur la même course et dans les mêmes conditions. Cela signifie que pour chacun des athlètes, tous les paramètres doivent être identiques : les entraînements précédents, le sommeil, l'alimentation, le coté psychologique etc. Or cela demande un réel cadre de recherche et je ne dispose pas d'assez de temps ni de moyens pour mener une étude de cette ampleur. Cependant c'est une étude qui serait vraiment intéressante à mener.\\

Un autre besoin qui a été évoqué est celui de pouvoir prédire le pic de forme. Le pic de forme est le moment où notre potentiel est porté à son maximum. J'ai réfléchi à ce sujet et j'en suis venu au fait que pour pouvoir prédire un pic de forme, il faut tout d'abord comprendre les courses et entraînements précédents. Cependant, mise à part l'analyse des différents temps de passage et du chronomètre final, rien ne nous permet de savoir ce qu'il se produit dans notre corps. C'est donc de ce besoin que ma problématique à émergé. En effet, lorsque nous serons capable de comprendre entièrement nos performances, nous pourrons par la suite prédire nos pics de forme. Cela permettra également aux entraîneurs d'adapter et d'optimiser leurs entraînements. Par exemple, en fonction des résultats, l'entraîneur pourra décider de repartir sur un nouveau cycle de travail pour améliorer ce qu'il manque à l'athlète au lieu de faire une compétition.
Cela pourrait donc être le sujet d'une nouvelle étude qui serait très utile dans la carrière d'un sportif. 


        
%%% Local Variables: 
%%% mode: latex
%%% TeX-master: "memoire"
%%% End: