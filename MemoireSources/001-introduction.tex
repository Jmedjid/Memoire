\chapter*{Introduction}
\addcontentsline{toc}{chapter}{Introduction}
\markboth{Introduction}{Introduction}
\label{chap:introduction}

Le sport est un domaine d'intérêt où les humains ont toujours essayé de repousser les limites. Chaque jour un record est brisé et la frontière de la performance est redéfinie.
Certaines découvertes ont malheureusement ouvert la porte à la tricherie, mais d’autres ont permis de comprendre comment obtenir de meilleurs résultats sans avoir recours à des substances interdites. Nous pouvons affirmer qu'aucune technologie, la plus avancée soit-elle ne peut remplacer le talent du sportif, l'intelligence de l’entraîneur et les longues heures d’entraînement. Cependant, la technologie peut aider les entraîneurs dans leurs analyses et les athlètes dans l'amélioration de leurs performances.\\

Partant de cette optique, mon choix de mémoire s'est tout naturellement orienté vers un sujet liant mes deux passions qui sont l'athlétisme et l'informatique, en m'intéressant à des outils utilisables dans le quotidien d'un athlète.
Ce sujet émane d'un réel besoin qui est celui de comprendre la performance. En effet, lorsque l'on termine une course, les seuls indicateurs de performance auxquels nous ayons accès sont le chronomètre, la distance ou encore le classement. Cependant ces résultats ne sont que des nombres et ne signifient finalement rien. Nous ne savons en réalité pas du tout ce qu'il s'est passé dans notre organisme. L'analyse des données récoltées grâce à ces outils pourrait permettre de mieux comprendre la performance d'un athlète et de mettre en lumière des axes d'amélioration dans son quotidien ainsi que dans son entraînement. 
C'est donc la recherche de l'amélioration des performances qui anime l'analyse des composantes individuelles de la performance. En effet, quand l'athlète et l'entraîneur peuvent isoler des zones sur lesquelles se concentrer à l'entraînement, le résultat final est susceptible d'être amélioré.\\

Étant athlète de haut niveau et pratiquant le 400m haies, l'utilisation de tels outils pourrait réellement m'aider dans l'accomplissement de mon rêve et de l'objectif de ma vie : les jeux olympiques.\\

La problématique de mon travail sera : la technologie permet-elle de rendre visible la partie invisible de la performance ?\\

Afin de répondre au mieux à cette question, je définirai dans un premier temps la performance ainsi que les déterminants de celle-ci en athlétisme, que ce soient l'optimisation des entraînements, l'environnement social de l'athlète, l'entraînement invisible ou encore les cotés psychologiques et physiologiques.
Dans un deuxième temps, je m’intéresserai aux différents outils existants afin de mettre en lumière la partie invisible de la performance.
Enfin, dans un troisième temps, j'analyserai les données mise à disposition, à travers des expérimentations sur différents athlètes et moi même afin de pouvoir conclure et répondre au problème posé.



%%% Local Variables: 
%%% mode: latex
%%% TeX-master: "memoire"
%%% End: