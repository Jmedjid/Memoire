\chapter*{Introduction}
\addcontentsline{toc}{chapter}{Introduction}
\markboth{Introduction}{Introduction}
\label{chap:introduction}

Le sport est un domaine d'intérêt où les humains ont toujours essayé de repousser les limites. Chaque jour un record est battu et la frontière de la performance est redéfinie. \\

Certaines découvertes ont malheureusement ouvert la porte à la tricherie, mais d’autres ont permis de comprendre comment obtenir de meilleurs résultats sans avoir recours à des substances interdites.\\

C'est notamment le cas de la technologie qui tient aujourd'hui une place prépondérante dans le monde du sport. Elle est utilisée en permanence, que ce soit pour augmenter l'efficacité des matériaux ou encore pour analyser des performances, dans le but de courir toujours plus vite ou de sauter toujours plus haut.\\

Toutefois, bien que la technologie permette d'analyser la performance, celle-ci est tellement complexe que sa compréhension n'est généralement que partielle.\\

En effet, lorsque nous terminons une course, le seul indicateur de performance auquel nous ayons accès est la partie visible de la celle-ci, représentée par le chronomètre, la distance ou encore le classement. Cependant, ces résultats ne sont que des nombres et ne signifient finalement rien. Nous ne connaissons en réalité pas du tout les effets que cet effort a produit sur notre organisme. \\

L'objectif de cette étude est donc de répondre à ce besoin de compréhension de la performance en s'intéressant à divers outils utilisables dans le quotidien d'un athlète et qui permettraient de rendre visible la partie invisible de la performance.\\

L'analyse des données récoltées grâce à ces outils pourrait permettre de mettre en lumière des axes d'amélioration dans l'entraînement de l'athlète, mais aussi des stratégies à adopter lors de l’approche des grandes compétitions.\\

C'est donc la recherche de l'amélioration des résultats qui anime l'analyse des composantes individuelles de la performance. En effet, dés lors que l'athlète et l'entraîneur sont capables d'isoler des zones sur lesquelles se concentrer à l'entraînement, le résultat final est susceptible d'être amélioré.\\

Étant athlète de haut niveau et pratiquant les disciplines de 400m et 400m haies, l'utilisation de tels outils pourrait réellement m'aider dans l'accomplissement de mon rêve et de l'objectif de ma vie : les jeux olympiques.\\



%%% Local Variables: 
%%% mode: latex
%%% TeX-master: "memoire"
%%% End: