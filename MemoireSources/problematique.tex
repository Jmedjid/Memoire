\chapter*{Problématique}
\addcontentsline{toc}{chapter}{Problématique}
\markboth{Problématique}{Problématique}
\label{chap:problématique}

A la suite de cette réflexion, la problématique la plus pertinente qui a émergé est : \\

\vspace{10pt}

\textbf{La technologie permet-elle de rendre visible la partie invisible de la performance ?}\\

\vspace{10pt}

La technologie et notamment divers outils de mesure sanguine pourraient en effet permettre de mesurer certains facteurs de la partie invisible de la performance.\\  

L'ensemble des données collectées seraient ensuite analysées par des logiciels d'analyse de données afin de révéler différents aspects qui ne seraient pas visible au premier abord et notamment les effets de l'effort produit sur l'organisme. \\

L'analyse de la partie invisible de la performance pourrait ainsi devenir un moyen courant de contrôle et d'optimisation de l'entraînement, de récupération, mais aussi de compréhension des performances réalisées en compétition.\\

En fonction des résultats obtenus, l'entraîneur pourrait déterminer pourquoi l'athlète a été plus performant sur une course plutôt qu'une autre et ainsi adapter les entraînements pour qu'ils correspondent à la physiologie et aux besoins de chaque athlète.\\

En outre, lorsque nous serons capable de comprendre nos performances, nous pourrons peut être par la suite améliorer une partie de la performance, ce qui est l'objectif de tout athlète.




%%% Local Variables: 
%%% mode: latex
%%% TeX-master: "memoire"
%%% End: