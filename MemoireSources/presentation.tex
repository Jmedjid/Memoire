\chapter*{Présentation}
\addcontentsline{toc}{chapter}{Présentation}
\markboth{Présentation}{Présentation}
\label{chap:introduction}
%\minitoc


%en quoi votre sujet se différencie-t-il des sujets des années précédentes (par exemple depuis 2014) en M2 MIAGE classique ?
L'an dernier, le sujet d'étude de Camille Leuregans était de savoir si la technologie pouvait remplacer le regard humain (en l’occurrence celui de l’entraîneur) dans la technicité de l’entraînement. Elle en ai arrivé à la conclusion que la technologie est utile pour seconder l’œil expert du coach, mais n'est pas capable de le remplacer complètement. En effet, aucune intelligence artificielle n'a pour le moment la capacité de soutenir moralement les athlètes, de les encourager, de les pousser à se surpasser et même d’anticiper les stratégies de courses des concurrents, afin de choisir la meilleure stratégie de course. 
Je pourrais donc  terminer ce mémoire en me demandant quels sont les moyens humains d'amélioration de la performance qui pourraient être automatisable par les outils informatiques.\\

%qu’est ce qui existe ?
Il existe une multitudes d'outils informatiques permettant d'améliorer la performance sportive. Cependant, ils ne seront pas utilisés de la même façon en fonction des spécialités de l'athlète. Il faut donc tout d'abord définir les facteurs qui influencent la performance pour un coureur de 400m.
Ces facteurs sont :
\begin{itemize}
\item Techniques (qualité du geste, coordination, précision)
\item Physiologiques (sang, taux de lactate)
\item Tactiques (prise de décisions)
\item Physiques (capacités énergétiques, musculaires)
\item Médicaux (santé, alimentation, prévention des blessures)
\item Psychologiques (mental, confiance)\\

\end{itemize}

Sommeil : résultat garmin mise en relation avec la performance




Chacun de ces facteurs possède ses spécificités et nécessite un outil particulier pour être amélioré.

Il existe tout d'abord des outils permettant d'analyser la forme du sportif qui sont: les objets connectés, notamment \textbf{les montres connectés} qui permettent de recueillir une multitudes de données, \textbf{les capteurs},  \textbf{l'électromyogramme} et \textbf{les appareils de mesures physique} (ex: biodex). Les résultats fournis par ces appareils sont devenus les premiers éléments d'ajustement de l'entraînement. Les premières données à récolter concernent le métabolisme de l'athlète, comme la quantité d'oxygène expirée, la quantité maximale d’oxygène que l’athlète est capable de consommer par unité de temps (VO2 max), sa fréquence cardiaque maximale, la force dégagée, la vitesse de l'athlète et enfin sa puissance. Il est également nécessaire de mesurer des données extérieures comme le temps de réaction, la fréquence de mouvement, la taille de la foulée etc.

%qu’est ce qui pourrait permettre de résoudre le problème ? 
Une fois que la forme de l'athlète est évaluée, les logiciels d'analyses récupèrent l’ensemble des données collectées et les étudient afin d’évaluer l’intensité et l’efficacité de l’entraînement et de l'optimiser. Ces résultats peuvent être analysés et stockés afin de les comparer dans le temps et avec ceux d’autres athlètes. Cela permet d’évaluer les performances de différents sportifs et de mettre au point des stratégies lors de l’approche des grandes compétitions. Il existe également des outils qui permettent d'analyser la technique de l'athlète comme \textbf{la vidéo} qui peut aider les athlètes à s’améliorer en décomposant leurs gestes image par image. L'analyse des données permet également la réduction des blessures grâce à des outils statistique. Il existe même une \textbf{application} qui calcul le risque d'avoir une blessure en fonction de divers facteurs comme les signes vitaux,l'alimentation, la qualité du sommeil et les antécédents de blessures.

La visualisation grâce à \textbf{la réalité augmentée ou virtuelle} est également un outil très puissant qui permet de jouer sur plusieurs facteurs. Tout d'abord sur le facteur mental car l'athlète aura réalisé sa course des centaines de fois et sera plus confiant. Grâce au phénomène des neurones miroirs, elle permettra également d'agir sur le facteur physique car il a été prouvé que la visualisation fait travailler les mêmes zones du cerveau que le réel accomplissement du mouvement.La \textbf{Go Pro} permet en ce sens d'enregistrer la course telle qu'elle est vue par l'athlète. Combiné à un casque de réalité virtuelle, cela permet de pouvoir revoir une course comme si l'athlète la courait réellement alors qu'il peut être assis dans son canapé. Cela permet alors de produire un entraînement plus lourd, sans ressentir de fatigue ou bien le risque de blessure.


Enfin, il existe des outils qui s'adressent plus particulièrement aux entraîneurs comme des générateurs d'entraînement. \\

%aurez-vous accès à des données pour le faire (si applicable à votre sujet) ?
Pour répondre au mieux au problème posé, j'effectuerai plusieurs études.

Tout d'abord, je réfléchirai aux outils informatiques qui me permettrait d'améliorer mes capacités. Je sonderai également mon entraîneur et mon kinésithérapeute pour connaître les outils qu'ils pensent nécessaire à l'amélioration de la performance.

J'analyserais ensuite des données recueillies lors d'études réalisées auprès d'autres athlètes de haut niveau dans mon club d’athlétisme et j'expérimenterai moi même plusieurs outils et essayerai de proposer un protocole de recherche avec les résultats de mon étude.

Je vais également essayer de récolter des données à l'INSEP (Institut National du Sport, de l'Expertise et de le Performance) pour pouvoir compléter mon étude.\\

% que comptez vous proposer / expérimenter ? votre réalisation comprend-elle une part de développement (qu’il s’agisse d’une application en Java, d’un tableau de bord en Excel, de règles pour outil de BI, etc) ?
Concernant le développement, j’ai cherché une solution pour implémenter ma propre application afin d'analyser la performance et j’ai découvert que Garmin mettait à disposition un SDK nommé Connect IQ permettant à chacun développer des applications, watch faces et widgets compatibles avec les montres de la marque. Le langage utilisé par ConnectIQ s’appelle le Monkey C qui est un langage orienté objet et le développement peut se faire sous Eclipse ou même IntelliJ.
Etant donné que je possède une montre Garmin, je pense donc, si j'ai le temps, développer ma propre application, d'autant plus que celle ci peut ensuite être publiée sur l'app store de Garmin.

J'avais également en tête une idée de lunette qui pourrait projeter un hologramme d'une personne faisant le lièvre à un athlète. L'hologramme serait paramétré par l'utilisateur en lui indiquant l''allure à laquelle il doit courir, s'il doit juste être suivi ou bien dépassé etc. Les lunettes disposeraient  également d'un tableau de bord avec un certains nombres d'indicateurs comme le chronomètre, l'allure, la distance parcourue etc.

En faisant des recherches, j'ai trouvé que des lunettes de réalité augmentée pour les cyclistes nommées Everysight Raptor AR ont été dévoilées en 2017. Elles sont équipées d'un écran qui affiche des informations en temps réel  devant les yeux de l'utilisateur comme des indications de navigation GPS , l’heure, la distance parcourue, la vitesse, et le rythme cardiaque de l’utilisateur. 

Cela correspond à l'idée du tableau de bord que  je me faisait mais en course le fait d'avoir en plus un hologramme qui fait lièvre est vraiment très utile. Cependant l'implémentation d'une telle application requiert des compétences et du temps que je n'ai pas. Il sera donc peut être possible pour moi de seulement en réaliser une simulation virtuelle en prenant en considération tous les paramètres techniques à spécifier afin de valider l'idée.\\


La combinaison de tous ces outils peut réellement contribuer à améliorer la performance d'un athlète et l'aider à maximiser son potentiel.\\


        
%%% Local Variables: 
%%% mode: latex
%%% TeX-master: "memoire"
%%% End: