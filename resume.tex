\chapter*{Résumé}
\addcontentsline{toc}{chapter}{Résumé}
\markboth{Résumé}{Résumé}
\label{chap:resume}
%\minitoc



Les relations entre certains aspects de la partie invisible de la performance comme la lactatémie, la glycémie, la fréquence cardiaque et l'oxymétrie de pouls ont été étudié chez 3 athlètes à la suite de courses de 400 mètres. Les mesures ont étés effectuées dans les 5 minutes suivant l'achèvement de l'effort avec des outils spécifiques. Les données recueillies ont ensuite été analysées.. \\

La découverte majeure de cette étude à été révélée par l'analyse des valeurs de la lactatémie. \\

Chez les 3 sujets, l'écart de lactatémie entre la valeur avant la course et celle après était très fortement lié à la performance réalisée (r=-
-0.97 avec P <0,05). Plus cet écart était important, meilleure était la performance de l'athlète relativement à ces autres performances. Cela indique alors qu'une des clés de la performance au 400 mètres est d'être capable de produire de grandes quantités de lactate tout en ayant des valeurs de lactatémie très basse avant la course.\\

Parallèlement à cela, la modification de l'échauffement d'un des sujets à la suite de l'analyse de taux anormaux de lactatémie, lui a permis d'améliorer considérablement ses performances (d'environ 2 secondes).\\

Ces résultats suggèrent que l'analyse de la partie invisible de la performance et notamment de la lactatémie permet de mieux comprendre la performance et dans certains cas, de l'améliorer.

        
%%% Local Variables: 
%%% mode: latex
%%% TeX-master: "memoire"
%%% End: