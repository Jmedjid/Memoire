\chapter*{Contexte}
\addcontentsline{toc}{chapter}{Contexte}
\markboth{Contexte}{Contexte}
\label{chap:contexte}

Le thème de ce mémoire a été très facile à trouver car il s'est tout naturellement orienté vers un sujet liant mes deux passions qui sont l'athlétisme et l'informatique. Toutefois, la problématique à quant à elle été plus compliquée à formuler. Beaucoup de pistes étaient intéressantes à explorer, mais aucune d'entre elles ne résolvaient un réel problème, ou bien elles étaient trop compliquées à résoudre.\\

Mon tuteur, Mr Lom Hillah, m'a alors suggéré d'effectuer une étude de terrain auprès des athlètes de mon club d'athlétisme afin de connaître leurs besoins. \\

Beaucoup d'entre eux m'ont fait part de leur souhait de pouvoir améliorer leurs performances grâce à l'utilisation de la technologie et notamment de divers outils informatiques. Cet axe était également une de mes premières idées, mais en en faisant part à mon tuteur mais également à mon entraîneur, je me suis rendu compte que cela serait trop compliqué à étudier. En effet, pour pouvoir déterminer si un outil permet réellement d'améliorer une performance il faudrait le tester sur un échantillon d'athlètes réalisant la même discipline, dans les mêmes conditions. Cela implique que pour chacun des athlètes, tous les paramètres doivent être identiques : les entraînements précédents, le sommeil, l'alimentation, le coté psychologique etc. Or cela demande un réel cadre de recherche et je ne dispose pas d'assez de temps ni de moyens pour mener une étude de cette ampleur. \\

Un autre besoin qui a été évoqué est celui de pouvoir prédire le pic de forme, c'est à dire le moment où notre potentiel est porté à son maximum. J'ai réfléchi à ce sujet et j'en suis venu au fait que pour pouvoir prédire un pic de forme, mais également améliorer les performances, il fallait tout d'abord comprendre les courses et les entraînements précédents. Cela passe non seulement par l'étude de la partie visible de la performance c'est à dire les différents temps de passage ou encore le chronomètre final, mais aussi et surtout, par l'étude de sa partie invisible. Cette partie invisible de la performance est déterminée entre autres par différents aspects physiologiques comme la lactatémie, la glycémie, l'oxymétrie ou encore la fréquence cardiaque de l'athlète, mais n'est que très peu étudiée par les athlètes et entraîneurs.\\


%%% Local Variables: 
%%% mode: latex
%%% TeX-master: "memoire"
%%% End: